\section{Matlab files related to noise calculations}
\subsection{noiseAddContrib}
\label{sec:noiseAddContrib}
\index{noiseAddContrib@\textsf{noiseAddContrib}}
\begin{verbatim}
 NOISECONTRIB constructor function for a noise contribution

\end{verbatim}

\newpage
\subsection{noiseCompute}
\label{sec:noiseCompute}
\index{noiseCompute@\textsf{noiseCompute}}
\begin{verbatim}
 NOISECOMPUTE computation of the output noise spectral density and other 
    derived quantities.
 
   NOISE = noiseCompute(CONTRIBARRAY, OUTNAME, OUTTYPE, TF, FREQ) computes the
   power spectral density at the output of a circuit. This output is specified
   with a name (string OUTNAME), a type (string OUTTYPE) and a transfer
   function TF. This transfer function is a structure with the following three
   fields:  
     - dcVal: the value of the transfer function for s = jw = 0
     - poles: an array containing the value of the poles that should be
              taken into account.
     - zeros: an array containing the value of the zeros that should be
              taken into account.
   The total output power spectral density depends on different
   contributions, which are given in the array CONTRIBARRAY. Each contribution
   is a structure, with the datastructure that has been constructed using the
   function noiseContrib.
   The units of the resulting power spectral density are
   A^2/Hz or V^2/Hz, depending on the type of the output of interest. The power
   spectral density is computed as a function of frequency, which is specified
   by the argument FREQ. This is an array of frequency points.
   The results are returned as part of the structure NOISE.
 
   NOISE = noiseCompute(CONTRIBARRAY, OUTNAME, OUTTYPE, TF, FREQ, EQINNAME,
   EQINTYPE) computes in addition the equivalent input noise power spectral
   density at the input, which is specified by a string (input argument
   EQINNAME) and a type (string EQINTYPE), which is either 'v' or 'i'.
 
   NOISE = noiseCompute(CONTRIBARRAY, OUTNAME, OUTTYPE, TF, FREQ, EQINNAME,
   EQINTYPE, SOURCEINDEX) computes in addition the spot noise figure as a
   function of frequency. Here, SOURCEINDEX is the index in the array
   CONTRIBARRAY that corresponds to the noise source of the source impedance.
 
   The datastructure of the return value NOISE is a structure with the
   following fields:
 
   1. contrib
      This is an array of contributions: contrib(1), contrib(2),
      ... contrib(i), ... It is a copy of the argument CONTRIBARRAY, with, for
      each contribution contrib(i), the addition of the following fields: 
      - contrib(i).out.psd
        this is the contribution of the i-th noise source to the power spectral
        density of the noise at the output;
      - contrib(i).eqIn.psd
        this is the contribution of the i-th noise source to the power spectral
        density of the noise at the output, referred back to the input;
      - contrib(i).nfLin
        this is the contribution of the i-th noise source to the noise figure
        (linear scale, not in dB);
   2. freq
      This is an array of frequency points, describing the frequency band over
      which noise power spectral density and integrated noise are computed
   3. nContribs
      number of contributions
   4. out
      This is in turn a structure with the following fields:
       - name
         this is the name (= string) of the output quantity;
       - type
         this is the type (= string) of the output quantity. This is either 'v'
         (for a voltage) or 'i' (for a current);
       - psd
         this is an array with the same length as the field "freq", with, for
         every corresponding frequency point, the output noise power spectral
         density;
   5. eqIn
      This is a structure that is constructed if equivalent input noise is to
      be computed. The structure has the following fields:
       - name
         this is the name (= string) of the input signal
       - type
         this is the type (= string) of the input signal. This is either 'v'
         (for a voltage) or 'i' (for a current);
       - psd
         this is an array with the same length as the field "freq", with, for
         every corresponding frequency point, the input-referred noise power
         spectral density
       - rms
         square root of the input-referred noise power, which is the integral
         of the input-referred noise power spectral density over the frequency
         band of interest, which is specified with "freq"
   6. tf
      This is the transfer function, which is a structure with three fields:
       - dcVal: the value of the transfer function for s = jw = 0
       - poles: an array containing the value of the poles that should be
              taken into account.
       - zeros: an array containing the value of the zeros that should be
              taken into account.
   7. nfLin
      This field is computed when the noise figure is to be computed. It is an
      array with the same length as the field "freq", with for every
      corresponding frequency point the noise figure (linear scale)
 
   (c) IMEC, 2005
   IMEC confidential 
 

\end{verbatim}

\newpage
\subsection{noiseContrib}
\label{sec:noiseContrib}
\index{noiseContrib@\textsf{noiseContrib}}
\begin{verbatim}
 NOISECONTRIB constructor function for a noise contribution

\end{verbatim}

\newpage
\subsection{noiseEnterSource}
\label{sec:noiseEnterSource}
\index{noiseEnterSource@\textsf{noiseEnterSource}}
\begin{verbatim}
 NOISEENTERSOURCE constructor function for a noise source and its contribution 
     to the total noise 
 
    CONTRIB = NOISEENTERSOURCE(NAME, VALUE, TYPE, OUTNAME, OUTTYPE, TF) returns
    a structure CONTRIB, that corresponds to data of one single noise
    source. The structure CONTRIB has the following fields:
    1. source
       This has the following fields:
        - name
          This is a string, representing the name of the source (e.g. 'Mp1
          thermal')
        - val
          This is the value of the noise source. This can be a tabulated value
          (e.g. Mp1.di2_id) or an equivalent input noise voltage (then one
          should use Mp1.di2_id / Mp1.gm^2)
        - type
          This is a string, which is either 'flicker' or 'thermal'
    2. tf
       This is the transfer function from the noise source to the circuit
       output. This circuit output is specified by its name, which is the
       argument OUTNAME (a string, e.g. 'vout'), and its type OUTTYPE. The type
       is either 'v' (when the output of interest is a voltage) or 'i' (when
       the output of interest is a current).
    3. out
       This has two fields initially:
       - name
         This is the name of the circuit output (= OUTNAME, see above).
       - type
         This is the type of the circuit output ('v' or 'i', see above)
    When the noise power spectral density or noise figure is computed (see
    function noiseCompute), then extra fields are added to CONTRIB.
    - the field out gets as extra field "psd"
    - if the equivalent input noise power spectral density is computed with
      noiseCompute, then the field eqIn is added. See noiseCompute for more
      details.
    - if the spot noise figure is computed with noiseCompute, then the field
    "nfLin" is added. See noiseCompute for more details.

\end{verbatim}

\newpage
\subsection{noisePlot}
\label{sec:noisePlot}
\index{noisePlot@\textsf{noisePlot}}
\begin{verbatim}
 NOISEPLOT plotting of noise power spectral densities or spot noise figure
 
  HANDLE = noisePlot(NOISE, INOUT) returns a handle or an array of handles to a
  plot of the noise power spectral density at the output or referred to the
  input, or the spot noise figure as a function of frequency. The input
  argument is a structure NOISE, which contains a field "freq" as well as the
  power spectral densities of interest or the spot noise figure. A description
  of the structure NOISE is given in noiseCompute and noiseEnterSource. in, out, inout
  all is default,  0 means only the total, [1 2 3] means that only
  contributions #1, #2 and #3 are plotted. 

\end{verbatim}

\newpage
\newpage
