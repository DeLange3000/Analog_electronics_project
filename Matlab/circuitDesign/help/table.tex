\section{Matlab files related to tables of a MOS transistor}
\subsection{tableDisplay}
\label{sec:tableDisplay}
\index{tableDisplay@\textsf{tableDisplay}}
\begin{verbatim}
 TABLEDISPLAY view the parameters that are present in a table for a MOS transistor
 
    tableDisplay(TABLE) displays information contained in the table that has
    been constructed for a given transistor type.
    The following information is displayed:
      1) DIMENSIONS: minimum gate length and width, width for which the table
         has been constructed (so-called reference width), width below which
         narrow-channel effects become visible (so-called critical width).
      2) A list of the input variables and their ranges
      3) A list of operating point parameters that are stored in the table
          (gm, ids, cgs, vdsat), .... Some parameters scale proportionally to
          the transistor width (at least above the critical width), other ones 
          (e.g. vdsat, vth) do not.
 
    EXAMPLE :
 
      tableDisplay(N) 
 
 
   (c) IMEC, 2004
   IMEC confidential 
 

\end{verbatim}

\newpage
\subsection{tableInArray}
\label{sec:tableInArray}
\index{tableInArray@\textsf{tableInArray}}
\begin{verbatim}
 TABLEINARRAY getting the array of values of an input variable of a MOS table.
 
     VALUE = tableInArray(INPUTVAR, TABLE) returns the array
     of values of the input variable INPUTVAR for a given TABLE of a MOS
     transistor.
 
     EXAMPLE :
 
       vgsArray = tableInArray('vgs', N);
 
     See also tableInFinal, tableInInit, tableInValue, tableInStep,
     tableInLength.
 
 
   (c) IMEC, 2004
   IMEC confidential 
 

\end{verbatim}

\newpage
\subsection{tableInCheckRange}
\label{sec:tableInCheckRange}
\index{tableInCheckRange@\textsf{tableInCheckRange}}
\begin{verbatim}
 TABLEINCHECKRANGE check whether an input parameter is inside the validity range
 
    tableInCheckRange(INPUTVAR, VALUE, TABLE) tests whether the VALUE of a given
    input parameter INPUTVAR (specified as a string) is inside the range of
    validity, i.e. the range that is used to construct the given TABLE for a
    MOS transistor.
    The function signals an error if the VALUE is not inside the range of
    validity.
 
 
 
   (c) IMEC, 2004
   IMEC confidential 
 

\end{verbatim}

\newpage
\subsection{tableInFinal}
\label{sec:tableInFinal}
\index{tableInFinal@\textsf{tableInFinal}}
\begin{verbatim}
 TABLEINFINAL getting the final value of an input variable of a MOS table.
 
     VALUE = tableInFinal(INPUTVAR, TABLE) returns the last element of the array
     of values of the input variable INPUTVAR for a given TABLE of a MOS
     transistor.
 
     EXAMPLE :
 
       vgsFinal = tableInFinal('vgs', N);
 
     See also tableInInit, tableInArray, tableInValue, tableInStep,
     tableInLength.
 
 
   (c) IMEC, 2004
   IMEC confidential 
 

\end{verbatim}

\newpage
\subsection{tableInInit}
\label{sec:tableInInit}
\index{tableInInit@\textsf{tableInInit}}
\begin{verbatim}
 TABLEININIT getting the initial value of an input variable of a MOS table.
 
     VALUE = tableInInit(INPUTVAR, TABLE) returns the first element of the array
     of values of the input variable INPUTVAR for a given TABLE of a MOS
     transistor.
 
     EXAMPLE :
 
       vgsInit = tableInInit('vgs', N);
 
     See also tableInFinal, tableInArray, tableInValue, tableInStep,
     tableInLength.
 
 
   (c) IMEC, 2004
   IMEC confidential 
 

\end{verbatim}

\newpage
\subsection{tableInLength}
\label{sec:tableInLength}
\index{tableInLength@\textsf{tableInLength}}
\begin{verbatim}
 TABLEINLENGTH getting the number of values of an input variable of a MOS table.
 
     N = tableInLength(INPUTVAR, TABLE) returns the number
     of values of the input variable INPUTVAR for a given TABLE of a MOS
     transistor.
 
     EXAMPLE :
 
       nVgs = tableInLength('vgs', N);
 
     See also tableInInit, tableInFinal, tableInArray, tableInValue, tableInStep,
     tableInLength.
 
 
   (c) IMEC, 2004
   IMEC confidential 
 

\end{verbatim}

\newpage
\subsection{tableInNames}
\label{sec:tableInNames}
\index{tableInNames@\textsf{tableInNames}}
\begin{verbatim}
 TABLEINNAMES input variables for the table of a MOS transistor
 
    CELLARRAY = tableInNames(TABLE) returns an array of strings that 
    are the names of the input variables of a given TABLE of intrinsic 
    operating point parameters of a MOS
    transistor. The order of these strings also corresponds to the order of the
    indices in the table values. E.g. when 'vgs' is the second element in the
    resulting CELLARRAY, then this also means that the second index of the
    table corresponds to vgs.
 
    EXAMPLE:
 
      stringArray = tableInNames(N);
 
 
   (c) IMEC, 2004
   IMEC confidential 
 

\end{verbatim}

\newpage
\subsection{tableInStep}
\label{sec:tableInStep}
\index{tableInStep@\textsf{tableInStep}}
\begin{verbatim}
 TABLEINSTEP step between two adjacent values of an input variable of a MOS table.
 
     VALUE = tableInStep(INPUTVAR, TABLE) returns the value of the step
     of the input variable INPUTVAR for a given TABLE of a MOS
     transistor. This step can be positive or negative. When the step is not
     constant, then the function returns NaN.
 
     EXAMPLES :
 
       vgsStep = tableInStep('vgs', N);
       lgStep = tableInStep('lg', N);
 
        This second example will return NaN.
 
     See also tableInInit, tableInFinal, tableInArray, tableInValue, tableInStep,
     tableInLength.
 
 
   (c) IMEC, 2004
   IMEC confidential 
 

\end{verbatim}

\newpage
\subsection{tableInValue}
\label{sec:tableInValue}
\index{tableInValue@\textsf{tableInValue}}
\begin{verbatim}
 TABLEINVALUE gets a particular value from the array of values 
 of an input variable of a MOS table of intrinsic operating point parameters.
 
     VALUE = tableInValue(INPUTVAR, INDEX, TABLE) returns the element with
     index INDEX in the array of values of the input variable INPUTVAR in a
     given TABLE of a MOS transistor.
 
     EXAMPLE :
 
       vgsInter = tableInValue('vgs', 5, N);
 
     See also tableInFinal, tableInInit, tableInArray, tableInStep,
     tableInLength.
 
 
   (c) IMEC, 2004
   IMEC confidential 
 

\end{verbatim}

\newpage
\subsection{tableLmin}
\label{sec:tableLmin}
\index{tableLmin@\textsf{tableLmin}}
\begin{verbatim}
 TABLELMIN minimum channel length of a MOS transistor
 
    LMIN = tableLmin(TABLE) returns the value of the minimum
    channel length that is allowed for the transistor type that is specified by
    the given TABLE of intrinsic operating point parameters of a MOS transistor.
 
    EXAMPLE :
 
      lmin = tableLmin(N);
 
    See also tableWmin, tableWcrit
 
 
   (c) IMEC, 2004
   IMEC confidential 
 

\end{verbatim}

\newpage
\subsection{tableMaxVdd}
\label{sec:tableMaxVdd}
\index{tableMaxVdd@\textsf{tableMaxVdd}}
\begin{verbatim}
 TABLEMAXVDD retrieving the maximally allowed VDD for a MOS transistor
 
    VALUE = tableMaxVdd(TABLE) returns the value of the maximally allowed VDD for
    a MOS transistor that is specified by a given TABLE of intrinsic
    operating point parameters of a MOS transistor.
 
    EXAMPLE :
 
      vdd = tableMaxVdd(N);
 
 
   (c) IMEC, 2004
   IMEC confidential 
 

\end{verbatim}

\newpage
\subsection{tableModelName}
\label{sec:tableModelName}
\index{tableModelName@\textsf{tableModelName}}
\begin{verbatim}
 TABLEMODELNAME model name of MOS transistor with which the table values 
  are computed.
 
    STRING = tableModelName(TABLE) returns the name of the MOS model with which
    the values in the given TABLE have been computed.
 
    EXAMPLE :
 
      modelname = tableModelName(N);
 
 
   (c) IMEC, 2004
   IMEC confidential 
 

\end{verbatim}

\newpage
\subsection{tableModelParam}
\label{sec:tableModelParam}
\index{tableModelParam@\textsf{tableModelParam}}
\begin{verbatim}
 TABLEMODELPARAM getting the value of a model parameter from a MOS table
 
    VALUE = tableModelParam(PARAMETER, TABLE) retrieves from the data of a given
    MOS table TABLE the numerical value of a model parameter that is specified
    with the string PARAMETER. No unit conversion is performed, which means
    that modelparameters that are not specified in S.I. units are returned in
    non-S.I. units.
 
    EXAMPLE :
 
       M1.cjsw = tableModelParam('cjsw', N);
 
     See also tableModelParamNames, tableModelName
 
 
   (c) IMEC, 2004
   IMEC confidential 
 

\end{verbatim}

\newpage
\subsection{tableModelParamNames}
\label{sec:tableModelParamNames}
\index{tableModelParamNames@\textsf{tableModelParamNames}}
\begin{verbatim}
 TABLEMODELPARAMNAMES list of all names of model parameters 
 
    RESULT = tableModelParamNames(TABLE) returns a cell array of strings. This cell
    array contains the names of all model parameters that have been defined in
    the given TABLE of a MOS transistor.
 
    EXAMPLE :
 
     modelParamNames = tableModelParamNames(N);
 
    See also tableModelParam, tableModelName
 
 
   (c) IMEC, 2004
   IMEC confidential 
 

\end{verbatim}

\newpage
\subsection{tablePre18}
\label{sec:tablePre18}
\index{tablePre18@\textsf{tablePre18}}
\begin{verbatim}
  preprocessing of the tables REG018N and REG018P of UMC 0.18um CMOS
 
   (c) IMEC, 2004
   IMEC confidential 
 

\end{verbatim}

\newpage
\subsection{tablePreprocess}
\label{sec:tablePreprocess}
\index{tablePreprocess@\textsf{tablePreprocess}}
\begin{verbatim}
 TABLEPREPROCESS preprocessing of a table of parameters of a MOS
 transistor.
 
   TABLEPREPROCESS(MOSTABLE, WMIN, LMIN, WREF, WCRIT, VDDMAX, MODELNAME,
   MODELSTRUCT, TECHNAME) changes a given raw table MOSTABLE, such that it is
   uniform for every technology. For example, the generated raw table can be 
   different depending on the MOS model: for a BSIM model, Spectre
   simulations generate different operating point parameters 
   (and sometimes with different names), than for a MM11 model.
   
   The following fields are added to the MOSTABLE
     - Input
     - Model. This contains several model parameters, namely the ones that 
       are needed for computations in existing functions, such as tox, VTO
       (if existing, does not exist in MM11), all parameters necessary
       to calculate junction capacitors for bulk CMOS, ...
       The model parameters that will be put in this field are the ones
       that correspond to a field of the structure MODELSTRUCT that is
       given as an argument to this function. For example, if MODELSTRUCT
       has a field MODELSTRUCT.tox with value 2e-9, then a field 
       MOSTABLE.Model.tox will be created and it has a value of 2e-9.
     - extra fields of MOSTABLE.Info: minimum allowed channel length in meters 
       and width in meters (= arguments LMIN and WMIN), reference width in
       meters (argument WREF), critical width in meters (argument WCRIT), 
       maximum allowed power supply voltage in Volts (argument VDDMAX),
       name of the model (argument MODELNAME, e.g. 'MOS Model 11') and name
       of the technology (argument TECHNAME, e.g. 'IMEC CMOS 90 nm')
 
   (c) IMEC, 2004
   IMEC confidential 
 

\end{verbatim}

\newpage
\subsection{tableTechName}
\label{sec:tableTechName}
\index{tableTechName@\textsf{tableTechName}}
\begin{verbatim}
 TABLETECHNAME returns the name of the technology for which the table has
   been made.
 
   (c) IMEC, 2004
   IMEC confidential 
 

\end{verbatim}

\newpage
\subsection{tableTechNode}
\label{sec:tableTechNode}
\index{tableTechNode@\textsf{tableTechNode}}
\begin{verbatim}
 TABLETECHNODE returns the feature size (as a number in meters) of the ...
     technology for which the table has been made.
 
   EXAMPLE:
   for a table corresponding to a 90 nm technology:
 
   tableTechNode(N)
 
   returns 9e-8
 
   (c) IMEC, 2005
   IMEC confidential 
 

\end{verbatim}

\newpage
\subsection{tableTechSource}
\label{sec:tableTechSource}
\index{tableTechSource@\textsf{tableTechSource}}
\begin{verbatim}
 TABLETECHSOURCE returns the name of the company, research center, ... from 
   which originates technology for which the table has been made.
 
   (c) IMEC, 2005
   IMEC confidential 
 

\end{verbatim}

\newpage
\subsection{tableType}
\label{sec:tableType}
\index{tableType@\textsf{tableType}}
\begin{verbatim}
 TABLETYPE returns the type of a transistor (n or p)
 
    string = tableType(TABLE) returns a string 'n' or 'p' depending on
    whether the TABLE under consideration refers to an n-MOS or p-MOS
    transistor. 
 
 
 
   (c) IMEC, 2004
   IMEC confidential 
 

\end{verbatim}

\newpage
\subsection{tableValueWref}
\label{sec:tableValueWref}
\index{tableValueWref@\textsf{tableValueWref}}
\begin{verbatim}
 TABLEVALUEWREF retrieving the value of an intrinsic MOS parameter for 
 the reference width, directly from a given table.
                                                                           
    RESULT = mosIntValueWref(PARAM, TABLE, LENGTH, VGS, VDS, VSB) returns
    the value of the MOS operating point parameter PARAM 
    (specified as a string) for a given TABLE.   
    PARAM must be exist as a field of the given TABLE. 
    Possible names for PARAM can be found by running tableDisplay(TABLE).
 
    EXAMPLE :                                                                
 
       id = tableValueWref('ids', N, 0.18e-6, 0.7, vdd/2, 0)                            
   
    See also tableDisplay, mosIntValue, mosWidth, tableWref
    
 
  The value of the parameter is computed by multidimensional linear 
  interpolation in the argument TABLE. We programmed the interpolation ourselves.
  This is much more efficient than using the Matlab function "interpn".
 
   (c) IMEC, 2004
   IMEC confidential 
 

\end{verbatim}

\newpage
\subsection{tableWcrit}
\label{sec:tableWcrit}
\index{tableWcrit@\textsf{tableWcrit}}
\begin{verbatim}
 TABLEWCRIT width of a MOS transistor below which narrow-channel effects play a role 
 
    WCRIT = tableWcrit(TABLE) returns the value of the minimum
    channel width that is allowed for the transistor type that is specified by
    the given TABLE. This value is usually smaller than the reference width.
 
    EXAMPLE :
 
      wCrit = tableWcrit(N);
 
    See also tableLmin, tableWref
 
   (c) IMEC, 2004
   IMEC confidential 
 

\end{verbatim}

\newpage
\subsection{tableWIndepParamNames}
\label{sec:tableWIndepParamNames}
\index{tableWIndepParamNames@\textsf{tableWIndepParamNames}}
\begin{verbatim}
 TABLEWINDEPPARAMNAMES returns operating point parameters of a MOS that are 
 assumed to be width independent. 
 
    stringArray = tableWIndepParamNames(TABLE) returns a cell array of
    strings with the names of operating point parameters stored in the given
    TABLE, and that are independent of transistor width above a critical width.
 
    EXAMPLE :
 
     stringArray = tableWIndepParamNames(N);
 
    See also tableWcrit
 
 
   (c) IMEC, 2004
   IMEC confidential 
 

\end{verbatim}

\newpage
\subsection{tableWInverseParamNames}
\label{sec:tableWInverseParamNames}
\index{tableWInverseParamNames@\textsf{tableWInverseParamNames}}
\begin{verbatim}
 TABLEWINVERSEPARAMNAMES returns operating point parameters of a MOS that are 
  assumed to be inversely proportional to the transistor width. Currently, this is
  only the power spectral density of the 1/f noise current source between
  source and drain ("di2_fn").
 
    stringArray = tableWInverseParamNames(TABLE) returns a cell array of
    strings with the names of operating point parameters stored in the given
    TABLE, and that - above a critical width - are proportional to 
    (transistor width)^(-1)
 
    EXAMPLE :
 
     stringArray = tableWInverseParamNames(N);
 
    See also tableWcrit, tableWIndepParamNames
 
 
   (c) IMEC, 2004
   IMEC confidential 
 

\end{verbatim}

\newpage
\subsection{tableWmin}
\label{sec:tableWmin}
\index{tableWmin@\textsf{tableWmin}}
\begin{verbatim}
 TABLEWMIN minimum channel width of a MOS transistor
 
    WMIN = tableWmin(TABLE) returns the value of the minimum
    channel width that is allowed for the transistor type that is specified by
    the given TABLE. This value is usually smaller than the reference width.
 
    EXAMPLE :
 
      Wmin = tableWmin(N);
 
    See also tableLmin, tableWcrit
 
   (c) IMEC, 2004
   IMEC confidential 
 

\end{verbatim}

\newpage
\subsection{tableWref}
\label{sec:tableWref}
\index{tableWref@\textsf{tableWref}}
\begin{verbatim}
 TABLEWREF gets the MOS transistor width with which the table of that MOS 
 is computed.
 
    W = tableWref(TABLE) returns the width in micrometer of the transistors with
    which the values in the given TABLE have been computed.
 
    EXAMPLE :
 
       width = tableWref(N);
 
 
   (c) IMEC, 2004
   IMEC confidential 
 

\end{verbatim}

\newpage
\newpage
